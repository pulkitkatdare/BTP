\documentclass[11pt]{article}                % Comment this line out
                                                          % if you need a4paper
%%%%%% FOR TIKZ %%%%%%%%%
%
\usepackage{tikz}
\usetikzlibrary{shapes,arrows,chains,matrix,positioning,scopes,shadows,calc}
\usetikzlibrary{decorations.pathmorphing,decorations.pathreplacing,decorations.shapes,snakes}
%
%%%%%% END FOR TIKZ %%%%%%%%

%%% Todos
\newcommand{\todo}[1]{\vspace{5 mm}\par \noindent
\marginpar{\textsc{ToDo}} \framebox{\begin{minipage}[c]{0.95
\textwidth} \tt #1 \end{minipage}}\vspace{5 mm}\par}
%
\usepackage{/Users/ravibanavar/PBMacBook_26_05_09/TEX_myhelp/macros}
\usepackage{amsfonts}
\usepackage{amssymb}
% \usepackage{MnSymbol}
\usepackage{amsmath}
\usepackage{amsthm}
\usepackage{graphics}
\usepackage{multirow}
\usepackage{empheq}
\usepackage{rotating}
\usepackage{tikz}
%\usepackage{a4wide}

\title{A geometric approach to modeling and controlof a Variable Speed Control Moment Gyro (VSCMG)}
	\author{Ravi N. Banavar and Arjun Narayanan \\
	\address{Systems and Control Engineering, \\
        Indian Institute of Technology Bombay, \\  Powai, Mumbai 400076, India. \\
        {\tt\small banavar@sc.iitb.ac.in}}}

\begin{document}

%%%%%%%%%%%%%%%%%%%%%%%%%%%%%%%%% FOR TIKZ %%%%%%%%%%%%%%%%%%%%%%%%%
\tikzstyle{point}= [coordinate]
\tikzstyle{block} = [rectangle, draw, text width=1.5cm, text badly centered, node distance=2cm, minimum
height=1cm]
\tikzstyle{sum} = [circle, draw, node distance=2cm, minimum height=0.6cm]
\tikzstyle{arrow} = [draw, thick, -latex', shorten >= 0.4mm]
\tikzstyle{line} = [draw, thick]
\tikzstyle{bridge}=[draw,double=gray, double distance=4mm]
%%%%%%%%%%%%%%%%%%%%%%%%%%%%%%%%% END FOR TIKZ %%%%%%%%%%%%%%%%%%%%%%%%%

\maketitle

\begin{abstract}
%
	Internal actuation mechanisms o control a rigid body, 
	like momentum wheels and control moment gyros (CMGs),
	have found much usage in spacecrafts and robotics.

	The Variable Speed Control Moment Gyro (VSCMG) has been much studied
	 in the aerospace and control community. There have been two distinct schools of work - 
	 one from the 
	aerospace community and the other from the geometric mechanics community.  The former
	adopts the Newtonian approach to derive the equations of motion and then goes on to study 
	singularity issues and control law synthesis. 
	The latter adopt tools from geometric mechanics - 
	principle fiber bundles, energy-Casimir notions - to derive control laws, though restricted to 
	spin stabilization and not full attitude control. This paper attempts to make connections 
	with the two approaches.

\end{abstract}
%
\section{Modelling in a geometric framework}
%
	We first model a single variable-speed CMG using the geometric framework. We then 
	compare the equations obtained with those in SRJ which are highly cited in the 
	aerospace literature.
	%
	\subsection{ Configuration space and moments of inertia}
	%
          With reference to the single CMG model as seen in figure, we explain our notation as:
          \\
          Viewed as a rigid body with independent of the 
           %----------------------------
	     %
           The significant differences in our approach as compared to the one by Schaub {\et al} are:
           %
           \been
           %
           \item We use the Riemannian metric to express the kinetic energy and then the momentum map to
                    express the equations of motion
                    %
           \item The 
           %
           \een
           %

          The configuration space is $Q = \SO3 \times \Sone \times \Sone $. 
          The first element denotes the 
          attitude $R_s$ of the spacecraft with respect to fixed frame, the second denotes the degree of 
          freedom $\beta$ of the gimbal frame, the third denotes the degree of freedom 
          $\gamma$ of the 
          rotor about its  spin axis. For the purpose of later geoemtrical interpretation, we club
          $x \deff (\beta, \gamma)$.
          
          There are three rigid bodies involved here, 
          each having relative motion (rotation) about the other. Three frames of reference
          are chosen - the first is the 
          spacecraft, denoted by the subscript $s$, the second is the gimbal frame, denoted by the 
          subscript $g$, the third is the rotor frame, denoted by the subscript $r$.        %
          The moments of inertia of the homogeneous rotor and the gimbal in their respective 
          body frames are assumed to be
          %
          \begin{align}
             \Irotord=  \pmat{J_x & 0 & 0 \\ 0 & J_x &  0  \\  0  &  0  & J_z } \;\;\;
             \Igimbal = \pmat{I_t & 0 & 0 \\ 0 & I_g &  0  \\  0  &  0  & I_s }
          \end{align}
            Since the rotor is 
          assumed to be homogeneous and symmetric, its inertia is represented in the gimbal frame  and
            the combined gimbal-rotor inertia is rewritten as     
               %
          \begin{align}
                 \Igimrot= \pmat{(J_x + I_t)& 0 & 0 \\ 0 & (J_x + I_g) &  0  \\  0  &  0  & (J_z + I_s) }
          \end{align}
          %
          If the rotational transformation that relates the gimbal frame to the spacecraft frame is 
          given by $R_{\beta}$, where
          $\beta$ denotes the gimballing angle,
          then the gimbal-rotor inertia reflected in the spacecraft frame is
          %
           \begin{align}
          (\Igimrot)_s \deff R_{\beta}  \Igimrot R_{\beta}^T
           \end{align}
	     %
	     Here the subscript $s$ denotes the spacecraft frame.
\subsection{Group action and kinetic energy}
	%	     
	     The action of the $\SO3$ group  on $Q$ is given by 
	     %
	     \begin{align}
	           \SO3 \times Q \rta Q \;\;\;\;\;\;   (M , (R_s, \beta, \gamma))  \rta (MR_s, \beta, \gamma)
	     \end{align}
	     %
	     and the tangent lifted action is given by 
	     %
	     \begin{align}
	           T \SO3 \times TQ \rta TQ \;\;\;   (v_{R_s}, v_{\beta}, v_{\gamma})  \rta 
	            (Mv_{R_s}, v_{\beta}, v_{\gamma})
	     \end{align}
	     %
	     If $\Omega_s = R_s^T \dot{R}_s$ denotes the angular velocity of the satellite in its
	     body frame, then the kinetic energy is given by
	      %
	      \begin{align}
	           \frac{1}{2} \inprod{  \Omega_s}{   \Is  \Omega_s}
	      \end{align}    
	      and the kinetic energy of the gimbal-rotor unit is given by
	     %
	     \begin{align}
	      \frac{1}{2} \inprod{R_{\beta}^T \Omega_s + \pmat{0 \\ \dot{\beta} \\ \dot{\gamma}} }
	                         {\Igimrot  [ R_{\beta}^T \Omega_s + \pmat{0 \\ \dot{\beta} \\ \dot{\gamma}} ] }
	      \end{align}
	      %
		The total kinetic energy is
	      %
	      \begin{align}
	      %
	              \frac{1}{2} \inprod{  \pmat{\Omega_s \\ 0 \\ \dot{\beta} \\ \dot{\gamma}}   }
	                         { \Itot   \pmat{\Omega_s \\ 0 \\ \dot{\beta} \\ \dot{\gamma}}     }
	      %
	      \end{align}
	      %
	      where 
	      %
	       \begin{align}
	       %
	       \Itot(\beta)= \pmat{ (R_{\beta}  \Igimrot R_{\beta}^T + \Is)  &  R_{\beta} \Igimrot  \\  \Igimrot  
	                              R_{\beta}^T &  \Igimrot   }
	       %
	      \end{align} 
	               %------------------------------
	      %
	      \begin{claim}
	      %
	            Under the defined group action, the kinetic energy of the total system 
	            remains invariant.
	      %
	      \end{claim}      
	      %
	      \proof  \\
	      Straightforward.
	      \\
	      $\Box.$
	      %
\subsection{A Riemannian structure}
%
		The kinetic energy induced a metric on the configuration space $Q$ of the
		system, which enables us to impart a Riemannian structure
		to the system. The Riemannian metric $\riem$, 
	      is expressed as 
	      %
	        \begin{align}
	        %
	             \inprod{v_q}{w_q}_{\riem} = \riem (R_s, (\beta, \gamma)) ((R_s \hat{\Omega}_1 , (v_1, v_2)),
	                                                           ((R_s \hat{\eta}_2 , (w_1, w_2))       
	        \end{align}
	        %
	        \begin{align}
	                       =  \frac{1}{2} \inprod{  \pmat{R_s \hat{\Omega}_1 \\ 0 \\ v_1 \\ v_2 }   }
	                         { \Itot   \pmat{R_s \hat{\eta}_2 \\ 0 \\ w_1 \\  w_2 }     }                                                                             
	        %
	        \end{align}
		where $q = (R_s, (\beta, \gamma)) \in Q $ and $v_q = (R_s\hat{\Omega}_1, (v_1,v_2)), 
		w_q = (R_s\hat{\Omega}_2, (w_1,w_2)) \in T_qQ$. Here $\hat{\Omega}_1$ and
		$ \hat{\Omega}_2$
		belong to the Lie algebra $\so3$
		%
\subsection{Principle fiber bundle}
%
	      More geometric structure is present in the problem. The gimbal angle and the rotor angle 
	      could be viewed as variables in a {\it shape space (or base space)} and the rigid body 
	      orientation could be viewed as a variable in a {\it fiber space}, and with a few additional
	      requirements, the model is
	      amenable to a principal fiber bundle description. See \cite{} for more details on 
	      describing mechanical systems in a fiber bundle framework.
	      Based on the above model description, we identify the principle fiber bundle 
	      $(Q, B, \pi, G)$,
	      where $Q = \SO3 \times \Sone \times \Sone$, 
	      $B = \Sone \times \Sone $ and $\pi: Q \rta B$ is the bundle projection.
	       We now define
	      a few mechanical quantities on this fiber bundle. 
	      \bei
	      % 
	      \item 
	      The {\it infinitesimal generator} of the Lie algebraic 
	      element $\hat{\eta} \in \so3$ under the group action is the vector field 
	      %
	      \begin{align}
	      %
	      \hat{\eta}_{Q}(q) = \frac{d}{dt}|_{t=0} (\exp(\hat{\eta}t) R_s, (\beta, \gamma)) = 
	                                 (\hat{\eta}R_s, (0, 0))
	     %
	      \end{align}
%
		\item The {\it momentum map} $J : TQ \rta \so3^*$ is given by
	      %
	      \begin{align}
	      %
	      [J(q, v_q), \xi] = \inprod{ v_q   }{  \hat{\xi}_Q(q) }_{\riem}
	      %
	      \end{align}
	      %
	      and 
	      since the kinetic energy is invariant under the action of the $\SO3$ group, we have
	      %
	       \begin{align}
	       %
	             \inprod{ v_q   }{  \hat{\xi}_Q(q) }_{\riem} = \inprod{ v_{(e,(\beta,\gamma))} }
	                                                                 {(R_s^T \hat{\xi} R_s,(0,0))  }_{\riem}
	       %
	       \end{align}
	       %
	       which yields
	       %
	       \begin{align}
	       %
	        [J(q, v_q), \hat{\xi}] =  \inprod{Ad_{R_s^T}^* [ (\Igimrot)_s + \Is) \Omega_s + R_{\beta} \Igimrot \pmat{ 0  \\  \dot{\beta}
	                                                      \\  \dot{\gamma} }   ] }{\xi}
	       \end{align}
	       %
	       Please note that $\inprod{\cdot}{\cdot}$ denotes the inner product, while 
	       $[\cdot, \cdot]$ denotes the primal-dual action of the vector spaces $\so3$ and
	       $\so3^*.$
	       Since the total spatial angular momentum of the system is constant, say $\mu$, 
	       the above expression
	       yields
	       %
	       \begin{align}
	       %
	       \mu = Ad_{R_s^T}^* [ (\Igimrot)_s + \Is) \Omega_s + R_{\beta} \Igimrot \pmat{ 0  \\  \dot{\beta}
	                                                      \\  \dot{\gamma} }   ]   \\
	                = 
	       \end{align}
	       %
	 	\item 
	 	   The {\it locked inertia tensor} at each point $q \in Q$ is the mapping 
	      %
	      \begin{align}  \mathbb I (q): \so3 \rta \so3^* \end{align}
	      %
	      and is defined as  
	      %
	      \begin{align}  [ \mathbb I(q) \eta, \xi ] = \riem (q)) ((\hat{\eta} R_s , (0,0)),
	                                                           (\hat{\xi} R_s , (0, 0))       
	      %
	      \end{align}
	      %
	      \item 
	      The {\it mechanical connection} is then defined as  the $\so3$-valued one-form	
	      %
	      \begin{align}
	      %  
	      		\alpha: TQ \rta \so3 \;\;\;\; (q, v_q) \rta \alpha(q, v_q) = {\mathbb I(q)}^{-1} J(q,v_q)
	      \end{align}
	      %
	      	\ei
      %
	                   
	       With the state-space as $X \deff (R_s, (\beta, \gamma)) = (R_s, x)$, where $x \deff (\beta, \gamma)$ 
	       and defining $\Itilde =  (R_{\beta}  \Igimrot R_{\beta}^T + \Is) $, the control inputs
	       (gimbal velocity and rotor spin)  at the kinematic
	       level as $u \deff \dot{x} = (\dot{\beta}, \dot{\gamma}) = \pmat{u_{\beta}  \\  
	       u _{\gamma} } $,
	       the affine-in-the-control system model is
	      %
	       \begin{align}
	       %
	             \dot{X} = f(X) + g(X) u 
	             %
	       \end{align}
	       %
	       where the drift and control vector fields are given by 
	       %
	       \begin{align}
	       %
	            f(X) = \pmat{R_s  \skew{ (  \Itilde^{-1} (Ad_{R_s}^* \mu))   }    \\  0 }  
	       %
	       \end{align}
	       %
	       \begin{align}
	       %
	            g_{\beta}(X) =  \pmat{- R_s  \skew{   (\Itilde^{-1}  (Ad_{R_\beta^T}^* (\Igimrot i_2))  }   
	             \\  
	                                     \pmat{1  \\  0}   }
	            \;\;\;\;\;
	            g_{\gamma}(X) =  \pmat{- R_s  \skew{ (\Itilde^{-1}  (Ad_{R_\beta^T}^* (\Igimrot i_3))  }     
	             \\  
	                                      \pmat{0  \\   1}     }
	       %
	       \end{align}
	       %
	       Here  $\skew{\cdot} : \R^3 \rta \so3$ is given by 
	       %
	       \begin{align}
	       %
	       \skew{(\psi_1, \psi_2, \psi_3)} \deff 
	       \pmat{ 0 &  - \psi_3  &  \psi_2  \\  \psi_3  &  0  &  -\psi_1  \\  
	                                                      - \psi_2  &  \psi_1  &  0 }
	        %
	        \end{align}
	        %
	        
\section{The dynamic model}
%
		To arrive at the dynamic model we proceed as follows.
		From the expression for the total momentum 
		%%
	       \begin{align}
	       %
	       \mu = Ad_{R_s^T}^* [ ((\Igimrot)_s + \Is) \Omega_s + R_{\beta} \Igimrot \pmat{ 0  \\  \dot{\beta}
	                                                      \\  \dot{\gamma} }   ]   \\
	                = 
	       \end{align}
	       %
	       we have
	       %
	       \begin{align}  \mu = R_s \pmat{ \Itilde & R_{\beta} \Igimrot}  \pmat{\Omega_s 
	       \\  \pmat{ 0  \\  \dot{\beta}  \\  \dot{\gamma} }  }
	       \end{align}
	       %
	       We split the momentum in to two components - one due to the gimbal-rotor unit and the other due to
	       the rigid spacecraft. Further, we assume an internal torque $\tau_b$ (in the gimbal-rotor fame)
	       acting on the gimbal and rotor unit. We then have, due to the principle of action and
	       reaction
	       %
	       \begin{align}
	       %
	       	\frac{d}{dt} (R_s \Is \Omega_s) = \underbrace{- R_s \tau_b}_{reaction}
		\;\;\;\; \;\;
		       \frac{d}{dt} ( R_s [ R_{\beta} \Igimrot R_{\beta}^T \Omega_s + R_{\beta} \Igimrot \dot{x} ] )
		       = \underbrace{R_s \tau_b}_{action}
	       \end{align}  
		We now simplify this expression to obtain more explicit equations, which we then 
		compare with certain other results.
		\[
		[ \hat{\Omega}_s \Itilde \Omega_s  +  \Itilde \dot{\Omega}_s ] 
		\]
		%
		\begin{align}
		%  
		= \hat{\Omega}_s R_{\beta} \Igimrot \dot{x} - \dot{\beta} R_{\beta} {\cal U} R_{\beta}^T
		\Omega_s - \dot{\beta} R_{\beta} \hat{i}_2 \Igimrot \dot{x} - R_{\beta} \Igimrot 
		\ddot{x}
		\end{align}
		%
		where ${\cal U} \deff \hat{i}_2 \Igimrot - \Igimrot \hat{i}_2$ is a symmetric matrix.
		
		
\section{The classical CMG modeling and analysis}		

		We now draw connections between the approach outlined in the previous sections
		with that of  the classical CMG modeling and analysis done in the Newtonian 
		framework in \cite{srj}, which is
		cited in much of the aerospace literature. We shall refer to this paper as the SRJ paper
		henceforth. We first relate the notation and then establish a 
		connection with the main equations of the SRJ paper. 
		
		The two primary variables in the SRJ paper and ours are related as follows:
		%
		\begin{center}
		\begin{tabular}{||  l  |  l   |  l   || }
		\hline
		Gimbal angle  & Our notation  - $\beta$  & SRJ notation - $\gamma$ \\
		\hline
		Rotor spin magnitude  & Our notation  - $\dot{\gamma}$  & SRJ notation - $\Omega$ \\
		\hline
		Satellite angular velocity  & Our notation  - $\Omega_s$  & SRJ notation - $\omega$ 
		\\
		\hline
		\end{tabular}
		\end{center}
		%
		The rotation matrix in the SRJ paper, relating the gimbal and spacecraft-body frame, 
		 is described in terms of three column vectors 
		of unit norm, $\{ \hat{g}_s, \hat{g}_{t}, \hat{g}_g \}$, where the subscripts $s, t$ and
		$g$ correspond to the {\it spin, transverse and gimbal} axes,  as
		%
		\begin{align} \pmat{| & | & |  \\  \hat{g}_s & \hat{g}_{t} & \hat{g}_g  \\ | & | & |  \\ }
		\end{align}
		%
		and further,
		\begin{align}
			\pmat{ \inprod{\hat{g}_s}{\omega} \\  \inprod{\hat{g}_{t}}{\omega}  \\ 
			 \inprod{\hat{g}_{g}}{\omega}    } = \pmat{  \omega_s  \\ \omega_t  \\  \omega_g }
		\end{align} 
		In our convention, the following correspondence holds: 
		%
		\begin{align} 
			R_{\beta} \longrightarrow 
			 \pmat{| & | & |  \\  \hat{g}_t & \hat{g}_{g} & \hat{g}_s  \\ | & | & |  \\ }
		\end{align}
		and
		%
		\begin{align}
		%
			R_{\beta}^T \Omega_s \longrightarrow 
			 \pmat{  \omega_t  \\ \omega_g  \\  \omega_s }
		%
		\end{align}
		The SRJ equation of motion (eqn. xxx) written partially in terms of our notation is
		%
		\begin{align}
		%  
			\Itilde \dot{\Omega}_s + \hat{\Omega}_s \Itilde \Omega_s = 
		\end{align}
		%
		\begin{align}
			- \hat{g}_s [ J_s (\ddot{\gamma} + \dot{\beta} \omega_t)  - (J_t - J_g) \omega_t 
			   \dot{\beta} ]
			   - \hat{g}_t [ J_s( \dot{\gamma} + \omega_s) \dot{\beta} - (J_t + J_g)
			   \omega_s \dot{\beta}  + J_s \dot{\gamma} \omega_g ] 
			   - \hat{g}_g [ J_g \ddot{\beta}  -  J_s \dot{\gamma} \omega_t ]
		%
		\end{align}
		while the RHS of the same equation in our notation is
		%
		\[
		     - [ \hat{\Omega}_s + \dot{\beta} \hat{i}_2 ]  R_{\beta} \Igimrot \dot{x}
		     - \dot{\beta} R_{\beta} ( \hat{i}_2 \Igimrot - \Igimrot \hat{i}_2 ) R_{\beta}^T \Omega_s
		     - R_{\beta} \Igimrot \ddot{x}
		\]
		%
		\[
		=  - \hat{g}_t [ (J_z + I_s) \dot{\gamma} \dot{\beta}  - (J_x + I_g) \dot{\beta} \omega_s
		   + (J_z + I_s) \dot{\gamma} \omega_g + (J_z + I_s)  - (J_x + I_t) \dot{\beta} \omega_s]
		   \]
		 \[
		 - \hat{g}_g [ (J_x + I_g) \ddot{\beta} - (J_z + I_s) \dot{\gamma} \omega_t ] 		   
		\] 
		%
		\[
		 - \hat{g}_s [ (J_z + I_s) \ddot{\gamma}  + (J_x + I_g) \dot{\beta} \omega_t  + 
		 ((J_z + I_s) - (J_x + I_t) ) \dot{\beta} \omega_t ]
		 \]
		
			\end{document}		
			
			
			
			
			
			
			
			
			
			
			
			
			
			
			
		
%\bibliographystyle{ieeetran}
%\bibliography{../../work/ref}

\end{document}
